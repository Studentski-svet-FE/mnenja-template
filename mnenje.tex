\documentclass[a4paper,12pt]{article}
\usepackage{predloga}

% Vhodni parametri
\newcommand{\datum}{6. 11. 2024}
\newcommand{\kandidat}{Lucifer Samael}
\newcommand{\pozicija}{Asistent}
\newcommand{\ocena}{\pozitivno}  % \pozitivno ali \negativno
\newcommand{\seja}{1}  % Katera redna seja
\newcommand{\predsednik}{David Hožič}  % Katera redna seja
%-----------------------------------


\begin{document}
\pagestyle{empty}
\includegraphics[height=2.637cm]{novlogo.pdf}
\vspace*{2cm}

\hfill Ljubljana, \datum

\section*{
    Študentsko mnenje o kandidatu\footnotemark\footnotetext{Moški samostalniki so uporabljeni nevtralno za moško in žensko slovnično obliko.}\\
    \color{Gray}\kandidat\footnotemark\footnotetext{Imena so v namen lažje interpretacije pisana v sklonu imenovalniku.}
}

\begin{center}
    za izvolitev v naziv \textbf{\pozicija}.
\end{center}

Študentski svet Fakultete za elektrotehniko je na osnovi anket sistema STUDIS na \seja. redni seji, dne \datum,\ 
podal \ocena\ študentsko mnenje za kandidata \textbf{\kandidat}.

% {AUTO_GEN}
Kandidat v predavalnico prihaja dobro pripravljen. Na vajah dobro razloži snov. Na predavanjih študentom
pomaga osvojiti kritična znanja, ki jih potrebuje za njegov uspešen študij.

\hfill

\hfill

\avtor{Predsednik ŠSFE}{\predsednik}
\end{document}
