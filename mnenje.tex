\documentclass[a4paper,12pt]{article}
\usepackage{predloga}

% Vhodni parametri
\newcommand{\datum}{20. 03. 2025}
\newcommand{\kandidat}{dr. Lucifer Samael}
\newcommand{\pozicija}{asistent}  % Pozicija (naziv) pisana z malo začetnico
\newcommand{\ocena}{\pozitivno}  % \pozitivno ali \negativno
\newcommand{\seja}{1}  % Katera redna seja
\newcommand{\predsednik}{David Hožič}  % Katera redna seja
\newcommand{\podrocje}{elektrotehnike}  % Področje izvolitve v rodilniku (e. g., elektrotehnike, matematike)
%-----------------------------------


\begin{document}
\pagestyle{empty}

\hfill Ljubljana, \datum

\section*{
    Študentsko mnenje o kandidatu\footnotemark\footnotetext{Moški samostalniki so uporabljeni nevtralno za moško in žensko slovnično obliko.}\\
    \color{Gray}\kandidat\footnotemark\footnotetext{Imena so v namen lažje interpretacije pisana v sklonu imenovalniku.}
}

\begin{center}
    za izvolitev v naziv \textbf{\pozicija} na področju \podrocje.
\end{center}

Študentski svet Fakultete za elektrotehniko je na osnovi anket sistema STUDIS na \seja. redni seji, dne \datum,\ 
podal \ocena\ študentsko mnenje za kandidata \textbf{\kandidat}.

{AUTO_GEN}

\hfill

\hfill

\avtor{Predsednik ŠSFE}{\predsednik}
\end{document}
